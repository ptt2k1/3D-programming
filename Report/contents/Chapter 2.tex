\section{Cài đặt chung}

\subsection{Thư viện}
Bài tập lớn này của chúng em sử dụng 5 thư viện sau:
\begin{itemize}
\item \textbf{three.js} (framework lập trình 3D cho Web trên webGL): thư viện JS sử dụng WebGL để vẽ 3D.
\item \textbf{dat.gui.js:} Thư viện này cho phép tạo một giao diện đơn giản để có thể thay đổi các biến trong code (Ví dụ: xoay đối tượng, điều chỉnh vị trí...).
\item \textbf{OrbitControls.js:} OrbitControls giúp ta xoay và pan một đối tượng ở giữa cảnh.
\item \textbf{OBJLoader.js:} Thư viện này dùng để đọc dữ liệu từ file định dạng .obj.
\item \textbf{MTLLoader.js:} Thư viện này dùng để đọc dữ liệu từ file định dạng .mtl.
\end{itemize}

\subsection{Camera}
\begin{itemize}
    \item \textbf{PerspectiveCamera} được dùng cho cả hai cảnh: ngoài trời và trong nhà.
\end{itemize}

\subsection{Light}
Chúng em sử dụng 3 nguồn sáng sau: 
\begin{itemize}
    \item \textbf{pointLight}
    \item \textbf{hemisphereLight}
    \item \textbf{ambientLight}
\end{itemize}

\subsection{Renderer}
\begin{itemize}
    \item \textbf{WebGLRenderer:} Renderer hiển thị các cảnh được tạo thủ công trong code bằng cách sử dụng thư viện WebGL. 
\end{itemize}