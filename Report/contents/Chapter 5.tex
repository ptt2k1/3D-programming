\section{kỹ thuật sử dụng}
Sử dụng thư viện \textbf{dat.gui.js} để thay đổi khung cảnh xung quanh
\subsection{Điều chỉnh khung hình}
Sản phẩm này gồm có 2 khung cảnh :
\begin{itemize}
    \item Ngoài trời
    \item Trong nhà
\end{itemize}
Vì thế khi thay đổi cảnh thì cần phải set lại camera.
\begin{lstlisting}[language = java]
 gui.add(controls, "switchCamera", {"out":0, "in":1}).onChange(function(e){
            switch (parseInt(e)){
                case 0: 
                    cam.position.set(0, 20, 110);
                    //cam.lookAt(scene.position);
                    break;
                case 1: 
                    cam.reset();
                    cam = cam1;
                    //cam.lookAt(scene1.position);
                    break;
            }
        });
\end{lstlisting}

Để dễ hơn trong việc lập trình, ở đây có sử dụng 2 camera và 2 scene khác nhau nên việc điều chỉnh camera ảnh hưởng đến scene. Khi đó ta cũng cần cập nhật lại scene ở phần \textbf{myRender} như sau:
\begin{lstlisting}[language = java]
if (controls.switchCamera == 0){
            renderer.render(scene, cam);
        }else{
            renderer.render(scene1, cam1);
        } 
\end{lstlisting}

Lúc này ta có thể lựa dễ dàng chọn khung hình mà mình muốn.

\subsection{Điều chỉnh background}
Với khung cảnh ngoài trời thì sẽ có 2 lựa chọn là ban ngày hoặc ban đêm vì thế nếu muốn thay đổi chúng ta cần:
\begin{itemize}
    \item Thay đổi texture của nền trời
    \item Thay đổi các loại màu đèn 
\end{itemize}
\begin{lstlisting}[language = java]
 gui.add(controls, "display", {"bright":0, "dark":1}).setValue(0).onChange(function(e){
            switch (parseInt(e)){
                case 0:
                    url_troi = textureLoader.load("btl/sky4.jpg");
                    sphereMat1.map = url_troi;
                    scene.remove(pointLight_dark);
                    scene.remove(ambientLight_dark);
                    scene.add(ambientLight);
                    scene.add(pointLight); 
                    break;
                case 1: 
                    url_troi = textureLoader.load("btl/star1.jpg");
                    sphereMat1.map = url_troi;
                    scene.remove(pointLight);
                    scene.remove(ambientLight);
                    scene.add(ambientLight_dark);
                    scene.add(pointLight_dark);
                    break;
            }
        });
\end{lstlisting}

\newpage
\section{Hướng phát triển sản phẩm}
Với mong muốn hoàn thiện sản phẩm này tốt hơn nữa thì nhóm chúng em có đề ra một số bước phát triển thêm như sau:
\begin{itemize}
    \item Liên kết 2 khung cảnh với nhau thông qua việc ấn vào cánh cửa của ngôi nhà
    \item Cải thiện phần khung nhà ở phần trong nhà tương ứng với ngôi nhà bên ngoài hơn.
    \item Thiết kế giao diện đẹp mắt hơn
    \item ...

\end{itemize}